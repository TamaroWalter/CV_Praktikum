\documentclass{article}
\usepackage[]{geometry} % Paket für die Seitenformatierung
\usepackage[ngerman]{babel} % deutsche Sprache
\usepackage{amsmath, amssymb, mathtools} % Mathematik
\usepackage{parskip} % Kein Einzug, stattdessen Abstand zwischen Absätzen
%Pfad zu Bildern

%-----------------------------------------
%           Allgemeine Einstellungen
%-----------------------------------------
\geometry{a4paper, total={170mm,257mm}, left=20mm, top=20mm, right=20mm, bottom=20mm}
\setlength\parindent{0pt}

\begin{document}

\begin{center}
    \Large{\textbf{Praktikum zu Computer Vision}}\\~\\
\end{center}
% Die richtige Datei einbinden
\begin{center} % Kopfbereich mit Infos zur Abgabegruppe
  \begin{tabular}{|rlp{4cm}rl|}
    \hline
    \textbf{Aufgabe:} & 01 &  & \textbf{1. Abgabepartner:} & Simon Schönhöffer \\
    \textbf{Gruppe:} & 14 &  & \textbf{2. Abgabepartner:} & Tamaro Walter  \\ \hline
  \end{tabular}
\end{center} 

\textbf{\\~\\Aufgabe 01:\\~\\}

Was auffällt:
\begin{enumerate}
  \item nutzt man keine Bildglättung gibt es viele kreisförmige Kanten in einem Bild
  \item Bildglättung bringt aber nicht so viel wie gedacht, selbst mit algorithmus der cv2 bibliothek sind ergebnisse eher Meh
  \item Eigener Bildglättungsalgorithmus ist eigentlich besser als der der cv2-bibliothek
  \item allgemein: Kantenerkennung funktioniert bei geraden objekten (ziegel, eckige häuser) besser als bei konturen (berge, bäume, runde bögen)
    \subitem ganz schlimm ist das uhu bild (42044), oder auch menschen (15011)
    \subitem Positivbeispiel: 17076, 28083
  \item welche parameter beeinflussen die edge-detection?
    \subitem die wahl von sigma für die Bildglättung
    \subitem die wahl der perzential für t_high und t_low bei der hysteresis_threshold_operation funktion.
      erhöht man t_high, werden kanten irgendwann aber nicht mehr erkannt. t_high sort aber auch dafür, dass weniger "kreise" erkannt werden
      erhöht man t_low, werden kreise eher vermieden. Es lohnt sich also t_low nahe an t_high zu setzen
\end{enumerate}